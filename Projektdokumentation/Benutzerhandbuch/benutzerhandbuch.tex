\section{Benutzerhandbuch}

\subsection{Stammdatenverwaltung}

\subsubsection{Die Stammdatenverwaltung aufrufen}

Die Stammdatenverwaltung rufen sie auf, indem Sie im Menü auf der rechten Seite auf das Schraubenschlüssel-Symbol klicken. Alternativ können sie auch das Menu ausklappen. Nun erscheint neben den Symbolen auch die Beschriftung.

\subsubsection{Räume bearbeiten}

In der Raum-Ansicht sehen sie auf der linken Seite eine Liste mit allen Räumen, die sie bereits definiert haben. Wenn sie einen neuen Raum anlegen möchten, Klicken sie auf den Button mit dem Plus-Symbol über der Raum-Liste. Nun öffnet sich auf der rechten Seite ein Formular. Tragen sie hier den Name des Raumes in das vorgesehene Textfeld ein. Klicken sie nun auf den Disketten-Button. Die Ansicht des Formulars ändert sich nun von der Bearbeitungs- auf die Leseansicht. 

Um einen Raum zu editieren, klicken sie auf den Bearbeiten-Button unter dem Formular. Dadurch wechselt die Lese- auf die Editieransicht. Das Editieren schließen sie ab, indem sie mit dem Disketten-Button Speichern.

Um einen nicht mehr benötigten Raum zu löschen, wählen sie in der Raumliste einen Raum an und klicken Sie auf den Mülltonnen-Button unter dem Eingabeformular.

Nach dem Speichern des Raumes können sie diesem Tische hinzufügen. Dies funktioniert auf die gleiche Art und Weiße wie bei einem Raum. Klicken sie auf das Plus-Symbol über der Tischliste, welche noch leer ist. Im Formular müssen sie einen Namen für den Tisch vergeben, sowie eine Anzahl von Sitzplätzen. Dies ist später im Betriebt wichtig, um einen schnellen Überblick über die Freien Tische und Plätze zu bekommen. Um den Tisch zu speichern klicken sie auf das Disketten-Symbol. Die Editieransicht wechselt dadurch auf die Leseansicht.

Um einen Tisch zu editieren, wählen sie einen Raum an und klicken sie dann auf das Edititeren-Symbol unter des Eingabeformulars. Das Editieren schließen sie durch Speichern mit einem klick auf den Disketten-Button ab.

Um einen nicht mehr benötigten Tisch zu löschen, wählen sie in der Tischliste einen Tisch an und klicken Sie auf den Mülltonnen-Button unter dem Eingabeformular.

\subsubsection{Warengruppen bearbeiten}

Um die Oberfläche für die Warengruppen aufzurufen, klicken sie in der Stammdatenverwaltung oben auf den Tab Warengruppen. Auf der linken Seite sehen sie eine Liste mit allen definierten Warengruppen. 

Wenn sie eine neue Warengruppe anlegen möchten, klicken sie auf den Plus-Button über der Warengruppenliste. Nun öffnet sie das Eingabeformular. Hier müssen sie einen Namen vergeben, sowie die übergeordnete Warengruppe angeben. So können sie eine Baumstruktur von Warengruppen erzeugen. Soll ihre Warengruppe keine Übergruppe haben, stellen sie den Schalter bei "Keine Übergruppe" auf "Ja".



\subsubsection{Waren bearbeiten}

\subsection{Buchungsverwaltung}

\section{Ein Ware auswählen und Buchen}

\section{Eine Buchung als Bezahlt oder Storniert markieren}

\subsection{Tagesübersicht}