\section{Aufgabenstellung}

\subsection{Funktionen}

\subsubsection{Stammdatenverwaltung}

Der Benutzer soll alle seine Ressourcen verwalten können. Hierfür soll es eine Stammdatenverwaltung geben, in der er seine Räume, Tisch, Waren und Warengruppen pflegen kann. Tische sollen Räumen untergeordnet sein, sowie Waren den Warengruppen. Außerdem soll der Benutzer bei den Warengruppen eine Baumstruktur anlegen können. So dass es zum Beispiel die Übergruppen "Getränke" und "Speißen" gibt und die anderen denen jeweils untergeordnet sind. 

Alle Ressourcen können angelegt, geändert und gelöscht werden. 

\subsection{Design}

Das Programm soll für eine Mobile Anwendungen, wie z.B. auf einem Tablet-PC, optimiert sein. Hierfür sind vor allem neben großen Schaltflächen auch eine gute Übersicht nötig. Der Nutzer darf in seinem Handeln nicht dadurch beeinträchtigt werden, dass er sich durch zu kleine Schaltfächen vertippt. Es soll ein dem User vertrautes Design verwendet werden, sodass es keine lange Eingewöhnungszeit gibt. 